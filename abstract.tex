\begin{abstract}
The performance characteristics of a software system depends to a significant extent on its configuration and workload. State-of-the-art performance modeling approaches either address configuration-dependent or workload-dependent performance behavior.  The relation and interaction of both factors and their influence on performance have not been systematically studied so far. Understanding to what extent configuration and workload---individually and combined---cause a software system’s performance to vary is key to understand whether performance models are generalizable, across different configurations and workloads. Assessing the impact and driving factors of such input sensitivity is key to develop strategies that obtain representative performance prediction models.

To shed light on this issue, we have conducted a \emph{systematic} empirical study, analyzing a multitude of configurations and workloads across a six software systems. We have obtained a substantial number of black-box performance measurements and enriched them with coverage data to assess whether and how configuration choices and workloads interact and shape software performance. 
We find that code coverage (i.e., \textit{what} code is executed) and code utilization (i.e., \textit{how} covered code is executed) are driving factors for workload-specific performance differences. Beyond code coverage testing, our findings motivate the use of dynamic code analyses to identify whether and in which way configuration options are sensitive to varying the workloads.
\end{abstract}
