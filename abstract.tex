\begin{abstract}
The performance of a software system depends on its configuration and the workload. 
State-of-the-art performance modeling approaches either address configuration-dependent or workload-dependent performance behavior.
The relation and interaction of both factors and their influence on performance, has not been \emph{systematically} studied so far.
%
Understanding to what extent configuration and workload---individually and combined---cause a software system's performance to vary is key to understand whether performance models are generalizable, across different configurations and wordloads. 
We would be able to determine whether a performance model is general and to what extent, whether we can efficiently transfer it to another setting, or whether need to entirely re-learn it with many configurations under a new workload.
%
To shed light on this issue, we have conducted a \textit{systematic} empirical study of a multitude of configurations and workloads across a selection of six configurable software systems. We have obtained a substantial number of black-box performance measurements and enriched them with statement coverage data to assess whether and how configuration choices and workloads interact and shape software performance. {\color{blue} Our findings indicate that .. }

\end{abstract}
