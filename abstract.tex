\begin{abstract}
Software performance depends on a software's configuration and workload. Such properties can be modeled using performance models learned from a sample set of observations. Existing work on performance modeling studies performance either under varying
by learning performance models for estimation, for instance, the response time of a given configuration, or, especially in practical performance testing, under varying workloads. The interaction and relation of both factors and its influence on performance, however, has not been \emph{systematically} studied so far.

Understanding to what extent configurability and workload in isolation and combined cause a software system's performance variation is key to determine whether performance models are generalizable to different application scenarios and settings. 
In essence, it is unknown whether, and if so, to what extent different workloads vary the utilization of configuration-dependent system functionality. From such knowledge, one can derive actionable strategies to, for instance, determine when we can keep a performance model, can efficiently transfer it to another workload, or need to entirely re-learn it with many configurations under a new workload.\todo{hier verlierst du die sicht des lesers. der weiß nichts von option-dependent code segements usw. -> vorsichtiger heran führen.}

On this behalf, we conduct a \textit{systematic} empirical study of a multitude of configurations and workloads across a selection of six configurable software systems. We enrich our black-box observations with statement coverage data to assess if and how configuration choices and workloads interact and shape software performance. {\color{blue} Our findings indicate that .. }

\end{abstract}
