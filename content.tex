\begin{abstract}
~ {\color{Orange}Configurable software system ...} Non-functional properties such as software performance often depend on configuration choices. Existing performance modeling approaches for configurable software systems do not take into account that further factors, most importantly, the workload. Performance models biased by the choice of the workload used during learning may not generalize to real-world use cases. {\color{Orange}Further description...}

In this paper, we explore the influence of different workloads on {\color{Orange}four} different non-functional properties across {\color{Orange}seven} configurable Java software systems. We augment performance measurements with dynamic code analysis to assess the feature coverage and representativeness of performance models in the presence of varying workloads. We devise a novel strategy to interpret performance models via lightweight code analysis and aim to raise awareness of multi-factor influences on software performance.
\end{abstract}


\section{Introduction}
\subsection*{Performance models are important}
Most software systems can be customized via configuration options to meet user demands. The selection of configuration options can enable desired functionality (features) or tweak non-functional aspects of a software system, such as improving performance or energy consumption. 
All too often, software performance bugs can be linked to configuration options~\cite{han_empirical_2016}. 
The relationship of configuration choices and their influence on non-functional properties has been extensively studied in recent years. The main focus was on the estimation non-functional properties, such as performance, for arbitrary configurations~\cite{dorn2020,siegmundPerformanceinfluenceModelsHighly2015,haDeepPerf2019,perfAL,guoVariabilityawarePerformancePrediction2013,sarkarCostEfficientSamplingPerformance,guo_2018_data,fourier_learning_2015,perLasso} or on finding configurations with near-optimal non-functional properties~\cite{nairUsingBadLearners2017,nairFlash18,ohFindingNearoptimalConfigurations2017}. 
The backbone of a performance estimate is a model that maps a given configuration to the estimated performance value. 
In general, performance prediction models are only useful to practitioners if predictions hold for a large number of application scenarios.

\subsection*{Performance models can be workload-biased}
Learning performance models for software performance usually relies on a training set of configuration-specific performance measurements. 
Typically, established approaches employ a single workload (e.g., a benchmark or test suite) to measure performance, which usually aims at emulating a real-world application scenario. 
As the collected measurements represent only the one application scenario, the resulting performance model is able to reliably estimate performance for this scenario. 
% rephrase sentence
{\color{Magenta}While existing work has demonstrated effective and efficient learning approaches, in essence, performance models obtained this way can target only workload-specific performance.}
The choice of a single workload can introduce bias as the predictive model is only as useful as the underlying workload resembles a wide range of use cases in the real world. 
Pereira et al. illustrate this fact with an analysis of the video encoder \textsc{x264} under different workloads~\cite{alves_sampling_2020}. 
They observed a wide variation in the performance behavior across a large set of different configurations and workloads. 
While this particular observation can be an outlier, in general, one cannot assume that performance behavior under a single workload is generalizable, and therefore consistent in every real-world setting.
In this vein, a recent exploratory study by Jamshidi et al.\cite{jamishidi_transfer_2017} illustrates that performance models of a configurable software system often exhibit differences in estimated performance influences when learned in different environments, including different versions, hardware setups, and workloads.
Influences across different environments remain largely congruent and differences only affect small numbers of configuration options, such that differences can be learned efficiently via transfer learning~\cite{jamshidi_transfer_gp_2017,jamshidi_learning_2018}. 
Nonetheless, under a single workload, we are left with performance models whose estimations are potentially inaccurate for different environments.

\subsection*{Representativeness}
Given not one, but a set of performance models based on different workloads, the question arises: Which performance model of the set is most general, whose estimations are more likely to \emph{represent} real-world scenarios? 
%rephrase
{\color{Magenta} The current perspective for performance models conceives these as predictors obtained from application-level, black-box observations.} 
To assess which model's estimations to trust most, we require information about whether and how different workloads interact with the software system under test. 
That is, we extend the black-box perspective of existing approaches~\cite{siegmundPerformanceinfluenceModelsHighly2015,haDeepPerf2019,perfAL,guoVariabilityawarePerformancePrediction2013,sarkarCostEfficientSamplingPerformance,guo_2018_data,fourier_learning_2015,perLasso} {\color{Magenta}with similarities from software testing} where the question usually is, what code is covered by a test case.
If the selection of a workload affecs the execution of performance-relevant code or determines how frequently code is executed, this can introduce substantial bias to a performance model. 
In addition, if such interactions affect option-specific code, the influence of configuration option is more susceptible to potential bias. {\color{Magenta}For example, ...}

As for a code section to contribute to performance requires its coverage, we argue that code coverage can serve as a lightweight, but effective indicator to assess potential workload bias and thus \emph{workload representativeness}. Along with coverage of feature-specific code sections, this indicator can extend to feature-specific performance influences.
To this end, we connect code coverage and estimated performance influences to make informed statements about 
which performance models and configuration options are more susceptible to workload bias and explore whether we can utilize code coverage testing as a means to select representative workloads to obtain robust and reliable performance models.

\begin{itemize}
	\item {\color{Orange} Approach/Experimente description }
	\item {\color{Orange} Results and findings }
\end{itemize}

\section{Background and Related Work}
\subsection{Performance Models for Configurable Software}
Configurable software systems are an umbrella term for any kind of software system that exhibits configuration options to customize functionality. 
While the primary purpose of configuration options is to select (categorical or binary options) and tune (numerical options) functionality, each configuration choice may also have implications on non-functional properties---be it intentional or unintentional. Finding configurations with optimal performance~\cite{nairUsingBadLearners2017,nairFlash18,ohFindingNearoptimalConfigurations2017} and estimating the performance for arbitrary configurations is an established line of research\cite{siegmundPerformanceinfluenceModelsHighly2015,haDeepPerf2019,perfAL,guoVariabilityawarePerformancePrediction2013,sarkarCostEfficientSamplingPerformance,guo_2018_data,fourier_learning_2015,perLasso}. 
Performance models are machine learning models that approximate non-functional properties, such as execution time or memory usage, as a function of software configurations $c \in C$, formally:

\begin{equation}
	\Pi: C \rightarrow \mathbb{R}
\end{equation}

Performance models are an approximation of real-world behavior based on training measurements and have been successfully obtained using a variety of machine learning techniques, including probabilistic programming\cite{dorn2020}, multiple linear regression\cite{siegmundPerformanceinfluenceModelsHighly2015}, classification-and-regression trees\cite{sarkarCostEfficientSamplingPerformance,guo_2018_data}, Fourier learning~\cite{fourier_learning_2015,perLasso}, and deep neural networks\cite{haDeepPerf2019,perfAL}. Most approaches share the perspective of treating a configurable software system as a black-box and model application-level performance agnostic of any code. 
Beyond the predominantly black-box perspective, recent work has incorporated feature location techniques to guide sampling effort towards relevant configuration options~\cite{velez_2020_configcrusher_jase,velez_comprex_2021} or model non-functional properties in a similar fashion, but at finer grain~\cite{weber_white_2021}.

\subsection{Configuration Sampling}
\begin{itemize}
	\item coverage sampling
	\item random sampling
	\item design of experiments
\end{itemize}

\subsection{Varying Workloads / Transfer Learning}
In case a changed workload is the main factor varied in an execution environment (e.g., hardware setup or workload), such bias can be explicitly learned~\cite{jamishidi_transfer_2017,jamshidi_learning_2018,jamshidi_transfer_gp_2017,ding_bayesian_2020}. Transformed performance prediction models, however, only rather adapt to a specific environment change exploiting similarities between corresponding performance prediction models. Contrary to transfer learning of performance prediction models, the influence of workloads on performance of configurable software systems has not been studied to the best of our knowledge.

\begin{itemize}
	\item performance spectra under different workloads~\cite{alves_sampling_2020}, 
	\item varying workloads for prediction~\cite{liao_2020_using_emse}
	\item relevance of micro-benchmarks \cite{laaber_emse_2021,grambow_peerj_2021}
\end{itemize}

\subsection{Associating Code and Features}
\begin{itemize}
	\item Static code analysis\cite{velez_2020_configcrusher_jase,lillack_2018_lotrack_tse,luo_2019_cova}
	\item Dynamic code analysis \cite{bell_phosphor_2014,velez_comprex_2021}
	\item Static program slicing and dynamic slicing (dicing)
	\item Software reconnaissance ~\cite{wong_integrated_2005,wilde_early_1996,agrawal_fault_1995,simmons_industrial_2006,chen_dynamic_nodate,sherwood_reducing_nodate}
\end{itemize}
\begin{itemize}
	\item comparison with \textsc{Comprex}~\cite{velez_comprex_2021}
	\item comparison with \textsc{SPLat}
\end{itemize}


\section{Problem Statement}
\subsection{Motivating Example}
\begin{comment}
\begin{lstlisting}[caption={Illustrative code example with one configuration option (\texttt{CACHING}) and workload-specific behavior.},label=lst:bener, language=Python]
CACHING = parse_options()

def calc_word_frequency(self, path: String)
	workload = open(path, 'r').read()
	workload_size = len(workload)
	frequencies = dict()

	# option-dependent
	if CACHING and path in self.cache:
		return self.cache[path]
		
	# workload-dependent
	if workload_size > 1024:	
		for word in workload.split(' '):
			if word not in frequencies: 
				frequencies[word] = 0
			frequencies[word] += 1
	
	# option- and workload-dependent
	if CACHING and workload_size > 65536:
		self.cache[path] = frequencies
					
	return frequencies
\end{lstlisting}
\end{comment}

\subsection{Exploring Workload-specific Differences}

\greybox{RQ1.1 \textit{Are there workload-specific differences in performance influence for configurable software systems?}}

\greybox{RQ1.2 \textit{Are there workload-specific differences in code coverage for configurations of a software system?}}

\greybox{RQ1.3 \textit{Do workload-specific differences in code coverage correspond to differences in performance influence?}}

\subsection{Code Coverage for Representativeness}

\greybox{RQ2.1 \textit{Can we use different workloads to identify option-specific code sections via superimposition?}}

\greybox{RQ2.2 \textit{Can we use code coverage or option-code coverage as a indicator for representativeness?}}

\begin{table*}
	\centering
	\caption{Experiment Characteristics}
	\begin{tabularx}{\linewidth}{lllrrrr}
		\toprule
		\textbf{Software System} &  \textbf{Application Type} & \textbf{Revision} & \textbf{ \#\,O} & \textbf{\#\,C} & \textbf{\#\,W}  \\
		\midrule
		\jumper & Audio encoder & 1.0.4 & 19 & 4\,196 & 6   \\
		
		\kanzi & File compressor & 1.9 & 24 & 4\,112 & 9 \\
			
		\dconvert & Image scaling & 1.0.0-alpha7 & 17 & 6\,764 & 12  \\
				
		\htwo & Embedded database & 1.4.200 & 16 & 1\,954  & 8  \\
		
		\batik & SVG rasterizer & 1.14 & 10 & 1\,919 &  11  \\
		
		\jadx & Java decompiler & 1.2.0 & 18 & 10\,502 & 9  \\
\bottomrule

\end{tabularx}\\
{\vspace{1mm}\textit{Abbreviations: \#\,O: No. of options, \#\,C: No. of configurations, \#\,W: No of. workloads}}

	\label{tab:subject_systems}
\end{table*}



\section{Study Setup}
\subsection{Subject Systems and Workloads}
\subsection{Operationalization}
\subsection{Results}

\section{Discussion}

\section{Threats to Validity}

\section{Conclusion}
