\documentclass[conference,review]{IEEEtran}
\IEEEoverridecommandlockouts
% The preceding line is only needed to identify funding in the first footnote. If that is unneeded, please comment it out.

\usepackage{blindtext}

\usepackage{listings}

% general imports
\usepackage{amsmath,amssymb,amsfonts}
\usepackage{cite}
\usepackage{algorithmic}
\usepackage{graphicx}

% layout and formatting
\usepackage{microtype}

% configure hyperlinks 
\usepackage{hyperref}
\hypersetup{
	colorlinks,
	linkcolor={PineGreen!90!white},
	citecolor={MidnightBlue!90!black},
	urlcolor={PineGreen!80!black}
}

% nice monospace font
\usepackage{sourcecodepro}

\usepackage{enumitem}


\usepackage{pdfcomment}


\usepackage{url}
\usepackage{float}

%nice font
%\usepackage{cochineal}


% configure subfigures
\usepackage{subcaption}

% configure tables
\usepackage{booktabs}
\usepackage{colortbl}
\usepackage[dvipsnames,table,xcdraw]{xcolor}
\usepackage{tabularx}

% table with footnotes
\usepackage{tablefootnote}

% fancy box for research questions, result summaries etc.
\usepackage[framemethod=TikZ]{mdframed}
\newcommand{\greybox}[1]{
	\begin{mdframed}[
		backgroundcolor=Black!10,
		linewidth=0.0pt,
		innerleftmargin=5pt,
		innertopmargin=5pt,
		roundcorner=1pt,
		]
		%\noindent
		#1
	\end{mdframed}
}


% remove indentioin for new paraghraphs
\setlength{\parindent}{0cm}

% utility imports
\usepackage{comment}

\lstset{framesep=2pt}
\lstset{
	language=Java, 
	basicstyle=\footnotesize\ttfamily,
	linewidth=0.48\textwidth,
	numbers=left,
	columns=flexible,
	numbersep=5pt,      % Abstand der Nummern zum Text
	tabsize=5,
	%breaklines=true,
	%frame=bt,
	showspaces=false,
	showtabs=false,
	xleftmargin=10pt,
	breakatwhitespace=true,
	framexleftmargin=10pt,
	framexrightmargin=5pt,
	framexbottommargin=4pt,
	showstringspaces=false,
	literate=%
	{Ö}{{\"O}}1
	{Ä}{{\"A}}1
	{Ü}{{\"U}}1
	{ß}{{\ss}}2
	{ü}{{\"u}}1
	{ä}{{\"a}}1
	{ö}{{\"o}}1
}

\newcommand{\cova}{\makebox[0pt][l]{\color{Goldenrod!15!white}\rule[-2pt]{1\linewidth}{10pt}}}
\newcommand{\covb}{\makebox[0pt][l]{\color{Magenta!15!white}\rule[-2pt]{1\linewidth}{10pt}}}
\newcommand{\covc}{\makebox[0pt][l]{\color{PineGreen!15!white}\rule[-2pt]{1\linewidth}{10pt}}}




\usepackage{xargs} 
\usepackage[colorinlistoftodos,prependcaption,textsize=tiny]{todonotes}
\newcommandx{\unsure}[2][1=]{\todo[linecolor=red,backgroundcolor=red!25,bordercolor=red,#1]{#2}}
\newcommandx{\change}[2][1=]{\todo[linecolor=blue,backgroundcolor=blue!25,bordercolor=blue,#1]{#2}}
\newcommandx{\info}[2][1=]{\todo[linecolor=OliveGreen,backgroundcolor=OliveGreen!25,bordercolor=OliveGreen,#1]{#2}}
\newcommandx{\improvement}[2][1=]{\todo[linecolor=Plum,backgroundcolor=Plum!25,bordercolor=Plum,#1]{#2}}
\newcommandx{\thiswillnotshow}[2][1=]{\todo[disable,#1]{#2}}

% Custom commands for research questions
\newcommand{\RQ}[2]{
	\hypertarget{rq:#1}{
		\vspace{0.5em}
		\bgroup
		\def\arraystretch{1.1}%  1 is the default, change whatever you need
		\begin{tabularx}{\linewidth}{p{0.12\linewidth}|p{0.8\linewidth}}
			%\rowcolor{PineGreen!10}
			RQ\,#1 & \textit{#2} \\
		\end{tabularx}
		\egroup
	}
}
\newcommand{\RQref}[1]{\hyperlink{rq:#1}{RQ\,#1}}

% Custom commands for subject systems
\usepackage{xspace}
\newcommand{\sosy}[1]{\textsc{\color{NavyBlue!95!white}#1}\xspace}
\newcommand{\jumper}{\sosy{jump3r}}
\newcommand{\htwo}{\sosy{h2}}
\newcommand{\hsqldb}{\sosy{hsqlsb}}
\newcommand{\kanzi}{\sosy{kanzi}}
\newcommand{\batik}{\sosy{batik}}
\newcommand{\dconvert}{\sosy{dconvert}}
\newcommand{\jadx}{\sosy{jadx}}

