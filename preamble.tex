\usepackage[english]{babel}

\usepackage{xcolor}
\usepackage{amsmath}
\usepackage{amsfonts}
\usepackage{tabularx}
\usepackage{algorithmic}
\usepackage{graphicx}
\usepackage{multirow}
\usepackage{cite}

% more compact enumerations / itemizes
\usepackage{paralist}

% configure hyperlinks 
\definecolor{citecolor}{HTML}{b8e0d2}
\definecolor{linkcolor}{HTML}{ffc09f}
\definecolor{urlcolor}{HTML}{f5e960}
\usepackage{hyperref}
\hypersetup{%
	colorlinks=false,
	urlbordercolor=urlcolor,
	linkbordercolor=linkcolor,
	citebordercolor=citecolor,
	pdfborder={0.8 0.8 0.8 },
}%

\usepackage{soul}
\usepackage{url}

% configure subfigures
\usepackage{subcaption}

% configure tables
\usepackage{booktabs}

% colorfule tables / rows
\usepackage{colortbl}

%
\usepackage{microtype}

% fancy box for research questions, result summaries etc.
\usepackage[framemethod=TikZ]{mdframed}
\newcommand{\greybox}[1]{
	\begin{mdframed}[
		backgroundcolor=cyan!5,
		linecolor=cyan!55,
		linewidth=0.5pt,
		innerleftmargin=4pt,
		innerrightmargin=4pt,
		innertopmargin=4pt,
		innerbottommargin=5pt,
		roundcorner=1pt,
		]
		%\noindent
		#1
	\end{mdframed}
}

\renewcommand\labelitemi{--}

% avoidorphaned words
\usepackage[all]{nowidow}


% Define thick line for tables, used in \RQ{} macro below
\usepackage{array}
\newcolumntype{?}{!{\vrule width 1pt}}

% Macro for research questions
\newcommand{\RQ}[2]{
	\vspace{0.5em}
	\noindent\hypertarget{rq:#1}{
		\bgroup
		\def\arraystretch{1.3}%  1 is the default, change whatever you need
		\begin{tabularx}{\linewidth}{p{0.1\linewidth}?p{0.79\linewidth}}
			\cellcolor{cyan!15} $\text{RQ}_\text{#1}$ &
			\cellcolor{cyan!5} \textit{#2} \\
		\end{tabularx}
		\egroup
	}
	\vspace{0.5em}
}

% Macro for references to \RQ{1} -> \RQref{1}
\newcommand{\RQref}[1]{\hyperlink{rq:#1}{$\text{RQ}_\text{#1}$}}

% Custom commands for subject systems
\usepackage{xspace}
\newcommand{\sosy}[1]{\textsc{\color{black}#1}\xspace}
\newcommand{\jumper}{\sosy{jump3r}}
\newcommand{\htwo}{\sosy{h2}}
\newcommand{\hsqldb}{\sosy{hsqlsb}}
\newcommand{\kanzi}{\sosy{kanzi}}
\newcommand{\batik}{\sosy{batik}}
\newcommand{\dconvert}{\sosy{dconvert}}
\newcommand{\jadx}{\sosy{jadx}}
\newcommand{\xz}{\sosy{xz}}
\newcommand{\lrzip}{\sosy{lrzip}}
\newcommand{\flac}{\sosy{FLAC}}
\newcommand{\vpnine}{\sosy{vp9}}
\newcommand{\xzwo}{\sosy{x264}}
\newcommand{\zdrei}{\sosy{z3}}

% define colors 
\definecolor{duplicatecheck}{HTML}{ffd3b1}
\definecolor{autocommit}{HTML}{b1c5ff}
\definecolor{nicegreen}{HTML}{79c900}
\definecolor{indigo}{rgb}{0.0, 0.25, 0.42}
\definecolor{cs-color-raw}{HTML}{ffa822}
\definecolor{lt-color-raw}{HTML}{95c9e2}
\definecolor{xmt-color-raw}{HTML}{86dbd4}
\definecolor{nmt-color-raw}{HTML}{fee4a7}

% mix colors
\colorlet{edited}{cyan!70!blue}
\colorlet{lt-color}{lt-color-raw!100!white}
\colorlet{xmt-color}{xmt-color-raw!100!white}
\colorlet{nmt-color}{nmt-color-raw!100!white}
\colorlet{xxx}{blue!30!cyan}
\colorlet{novelblue}{xxx!60!black}
\colorlet{novelblue2}{xxx!90!black}
