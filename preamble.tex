\usepackage{listings}

\newcommand{\xmark}{\ding{55}}

\usepackage{xcolor}
\usepackage{amsmath}
\usepackage{amsfonts}
\usepackage{tabularx}
\usepackage{algorithmic}
\usepackage{graphicx}
\usepackage{todonotes}
\usepackage{setspace}

% layout and formatting
\usepackage{microtype}
\usepackage{multirow}


% for emergencies
% suble = develpent
% moderate = serious
%\usepackage[subtle]{savetrees}

\usepackage[noadjust]{cite}

\usepackage{paralist}


% configure hyperlinks 
\usepackage{hyperref}
\hypersetup{%
	%colorlinks,
	linkcolor={brown!50!red},
	citecolor={blue},
	urlcolor={green!60!black}
}%

\usepackage{pdfcomment}

\usepackage{makecell}
\usepackage{soul}

\renewcommand\theadalign{bc}
\renewcommand\theadfont{\bfseries}
\renewcommand\theadgape{\Gape[0pt]}
\renewcommand\cellgape{\Gape[0pt]}

\usepackage{url}
\usepackage{float}

% configure subfigures
\usepackage{subcaption}
\usepackage{amsfonts}

% configure tables
\usepackage{booktabs}
\usepackage{colortbl}
\usepackage{blindtext}
% fancy box for research questions, result summaries etc.
\usepackage[framemethod=TikZ]{mdframed}
\newcommand{\greybox}[1]{
	\begin{mdframed}[
		backgroundcolor=nicegreen!10!white,
		linecolor=nicegreen!50!white,
		linewidth=2pt,
		innerleftmargin=5pt,
		innerrightmargin=5pt,
		innertopmargin=5pt,
		roundcorner=2pt,
		]
		%\noindent
		#1
	\end{mdframed}
}

\usepackage[framemethod=TikZ]{mdframed}
\newcommand{\challenge}[1]{
	\begin{mdframed}[
		backgroundcolor=79c900!15,
		linewidth=0.0pt,
		innerleftmargin=5pt,
		innertopmargin=5pt,
		roundcorner=3pt,
		]
		%\noindent
		#1ediuted
	\end{mdframed}
}
\renewcommand\labelitemi{---}

\lstset{framesep=2pt}
\lstset{
	language=Java, 
	basicstyle=\footnotesize\ttfamily,
	linewidth=0.48\textwidth,
	numbers=left,
	columns=flexible,
	numbersep=5pt,      % Abstand der Nummern zum Text
	tabsize=2,
	%breaklines=true,
	%frame=bt,
	showspaces=false,
	showtabs=false,
	xleftmargin=10pt,
	breakatwhitespace=true,
	framexleftmargin=10pt,
	framexrightmargin=5pt,
	framexbottommargin=4pt,
	showstringspaces=false,
	literate=%
	{Ö}{{\"O}}1
	{Ä}{{\"A}}1
	{Ü}{{\"U}}1
	{ß}{{\ss}}2
	{ü}{{\"u}}1
	{ä}{{\"a}}1
	{ö}{{\"o}}1
}


\definecolor{duplicatecheck}{HTML}{ffd3b1}
\definecolor{autocommit}{HTML}{b1c5ff}
\definecolor{nicegreen}{HTML}{79c900}

\definecolor{indigo}{rgb}{0.0, 0.25, 0.42}

\newcommand{\cova}{\makebox[0pt][l]{\color{duplicatecheck}\rule[-2pt]{0.92\textwidth}{8pt}}}
\newcommand{\covb}{\makebox[0pt][l]{\color{autocommit}\rule[-2pt]{0.92\textwidth}{8pt}}}
\newcommand{\covc}{\makebox[0pt][l]{\color{yellow!15!white}\rule[-2pt]{0.92\textwidth}{8pt}}}

% color
% Custom commands for research questions
\newcommand{\RQ}[2]{
	\hypertarget{rq:#1}{
		\vspace{0.5em}
		\bgroup
		\def\arraystretch{1.3}%  1 is the default, change whatever you need
		\begin{tabularx}{\linewidth}{p{0.12\linewidth}?p{0.75\linewidth}}
			\rowcolor{white!5!white}
			$\text{RQ}_\text{#1}$ & \textit{#2} \\
		\end{tabularx}
		\egroup
	}
}

%\colorbox{mentioned-color}{text}
\newcommand{\RQref}[1]{\hyperlink{rq:#1}{$\text{RQ}_\text{#1}$}}

\usepackage{comment}

% Custom commands for subject systems
\usepackage{xspace}
\newcommand{\sosy}[1]{\textsc{\color{blue!40!black}#1}\xspace}
\newcommand{\jumper}{\sosy{jump3r}}
\newcommand{\htwo}{\sosy{h2}}
\newcommand{\hsqldb}{\sosy{hsqlsb}}
\newcommand{\kanzi}{\sosy{kanzi}}
\newcommand{\batik}{\sosy{batik}}
\newcommand{\dconvert}{\sosy{dconvert}}
\newcommand{\jadx}{\sosy{jadx}}

\newcommand{\xz}{\sosy{xz}}
\newcommand{\lrzip}{\sosy{lrzip}}
\newcommand{\flac}{\sosy{FLAC}}
\newcommand{\vpnine}{\sosy{vp9}}
\newcommand{\xzwo}{\sosy{x264}}
\newcommand{\zdrei}{\sosy{z3}}

\definecolor{cs-color-raw}{HTML}{ffa822}

\definecolor{lt-color-raw}{HTML}{95c9e2}
\definecolor{xmt-color-raw}{HTML}{86dbd4}
\definecolor{nmt-color-raw}{HTML}{fee4a7}
\definecolor{edited}{HTML}{000000}

%\colorlet{cs-color}{cs-color-raw!28!white}
\colorlet{lt-color}{lt-color-raw!100!white}
\colorlet{xmt-color}{xmt-color-raw!100!white}
\colorlet{nmt-color}{nmt-color-raw!100!white}


\colorlet{xxx}{blue!30!cyan}
\colorlet{novelblue}{xxx!60!black}
\colorlet{novelblue2}{xxx!90!black}
\usepackage[english]{babel}

% thicc line
\usepackage{array}
\newcolumntype{?}{!{\vrule width 1pt}}

\usepackage{tikz}
\newcommand*\circled[1]{\tikz[baseline=(char.base)]{
		\node[shape=circle,draw,inner sep=0.7pt] (char) {#1};}}
	

% borrowed from https://tex.stackexchange.com/questions/12703/how-to-create-fixed-width-table-columns-with-text-raggedright-centered-raggedlef	
\usepackage{array}
\newcolumntype{L}[1]{>{\raggedright\let\newline\\\arraybackslash\hspace{0pt}}m{#1}}
\newcolumntype{C}[1]{>{\centering\let\newline\\\arraybackslash\hspace{0pt}}m{#1}}
\newcolumntype{R}[1]{>{\raggedleft\let\newline\\\arraybackslash\hspace{0pt}}m{#1}}
