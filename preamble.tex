\documentclass[conference,review]{IEEEtran}
\IEEEoverridecommandlockouts
% The preceding line is only needed to identify funding in the first footnote. If that is unneeded, please comment it out.

\usepackage{blindtext}

\usepackage{listings}

% general imports
\usepackage{amsmath,amssymb,amsfonts}
\usepackage{cite}
\usepackage{algorithmic}
\usepackage{graphicx}

% layout and formatting
\usepackage{microtype}

% configure hyperlinks 
\usepackage{hyperref}
\hypersetup{
	colorlinks,
	linkcolor={red!70!black},
	citecolor={blue!70!black},
	urlcolor={Orange!80!black}
}

\usepackage{sourcecodepro}

\usepackage{enumitem}

\usepackage{url}
\usepackage{float}
\usepackage{crimson}
%\usepackage[default]{comicneue}

% configure subfigures
\usepackage{subcaption}

% configure tables
\usepackage{booktabs}
\usepackage{colortbl}
\usepackage[dvipsnames,table,xcdraw]{xcolor}

% fancy box for research questions, result summaries etc.
\usepackage[framemethod=TikZ]{mdframed}
\newcommand{\greybox}[1]{
	\begin{mdframed}[
		backgroundcolor=Black!8,
		linewidth=0.1pt,
		innerleftmargin=5pt,
		innertopmargin=5pt,
		roundcorner=3pt,
		]
		%\noindent
		#1
	\end{mdframed}
}

% utility imports
\usepackage{comment}

\lstset{framesep=2pt}
\lstset{
	language=Python, 
	basicstyle=\footnotesize\ttfamily,
	linewidth=0.48\textwidth,
	numbers=left,
	columns=flexible,
	numbersep=5pt,      % Abstand der Nummern zum Text
	tabsize=5,
	%breaklines=true,
	%frame=bt,
	showspaces=false,
	showtabs=false,
	xleftmargin=10pt,
	breakatwhitespace=true,
	framexleftmargin=10pt,
	framexrightmargin=5pt,
	framexbottommargin=4pt,
	showstringspaces=false,
	literate=%
	{Ö}{{\"O}}1
	{Ä}{{\"A}}1
	{Ü}{{\"U}}1
	{ß}{{\ss}}2
	{ü}{{\"u}}1
	{ä}{{\"a}}1
	{ö}{{\"o}}1
}

\newcommand{\cova}{\makebox[0pt][l]{\color{NavyBlue!10!white}\rule[-2pt]{1\linewidth}{10pt}}}
\newcommand{\covb}{\makebox[0pt][l]{\color{NavyBlue!10!white}\rule[-2pt]{1\linewidth}{10pt}}}
\newcommand{\covc}{\makebox[0pt][l]{\color{NavyBlue!10!white}\rule[-2pt]{1\linewidth}{10pt}}}

\newcommand{\sosy}[1]{\textit{\color{RoyalPurple}#1}}


\usepackage{xargs} 
\usepackage[colorinlistoftodos,prependcaption,textsize=tiny]{todonotes}
\newcommandx{\unsure}[2][1=]{\todo[linecolor=red,backgroundcolor=red!25,bordercolor=red,#1]{#2}}
\newcommandx{\change}[2][1=]{\todo[linecolor=blue,backgroundcolor=blue!25,bordercolor=blue,#1]{#2}}
\newcommandx{\info}[2][1=]{\todo[linecolor=OliveGreen,backgroundcolor=OliveGreen!25,bordercolor=OliveGreen,#1]{#2}}
\newcommandx{\improvement}[2][1=]{\todo[linecolor=Plum,backgroundcolor=Plum!25,bordercolor=Plum,#1]{#2}}
\newcommandx{\thiswillnotshow}[2][1=]{\todo[disable,#1]{#2}}
